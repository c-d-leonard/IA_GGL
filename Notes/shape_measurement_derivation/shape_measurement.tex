\documentclass[onecolumn,amsmath,aps,fleqn, superscriptaddress]{revtex4}
 
\usepackage{amssymb}
\usepackage{amsmath}
\usepackage{epsfig}
\usepackage{subfigure}
\usepackage{mathrsfs}
\usepackage{longtable}
\usepackage{enumerate} 
\usepackage{multirow}
\usepackage{color}

\usepackage[usenames,dvipsnames]{xcolor}
\usepackage{hyperref}
\hypersetup{
    colorlinks = true,
    citecolor = {MidnightBlue},
    linkcolor = {BrickRed},
    urlcolor = {BrickRed}
}

\newcommand{\be}{\begin{eqnarray}}
\newcommand{\ee}{\end{eqnarray}}

\allowdisplaybreaks[1]
 
\begin{document}

\title{Intrinsic alignments from different shape measurement techniques}

\author{Danielle Leonard}

\maketitle

From Rachel's notes, we have:
\begin{equation}
\langle \delta_g^i \hat{g}^j\rangle(r_p) - \langle \delta_g^i \hat{g}^{\prime,j}\rangle(r_p) = (1-a_j)\langle \delta_g^i \gamma_{IA}^j\rangle(r_p)
\label{gen_2pt}
\end{equation}
where $i$ labels the lens bin, $j$ labels the source bin, and a prime indicates a different shape measurement to get the reduced shear $g$.

For the two-point functions, we could either use the averaged tangential shear about lens galaxies, or we could use $\Delta \Sigma$. I'm not sure which is best so I'm going to do the math for both below.

\section{Observable: averaged tangential shear}
Specify first to use the averaged tangential shear about a lens galaxy as our two-point function. All shears here should be assumed to be tangential. Write:
\begin{align}
\tilde{\gamma}(r_p) &= \frac{B(r_p) \sum\limits_{k}^{\rm lens} \tilde{w}_k \tilde{\gamma}_k}{\sum\limits_{k}^{\rm lens} \tilde{w}_k} = \frac{B(r_p) \sum\limits_{k}^{\rm lens} \tilde{w}_k \Sigma_{c,k}^{-1}\Delta\Sigma}{\sum\limits_{k}^{\rm lens} \tilde{w}_k}+\frac{B(r_p) \sum\limits_{k}^{\rm lens} \tilde{w}_k \gamma_{\rm IA}^k}{\sum\limits_{k}^{\rm lens} \tilde{w}_k} \nonumber \\ 
\tilde{\gamma}'(r_p) &= \frac{B^\prime(r_p) \sum\limits_{k}^{\rm lens} \tilde{w}^\prime_k \tilde{\gamma}_k}{\sum\limits_{k}^{\rm lens} \tilde{w}^\prime_k} = \frac{B^\prime(r_p) \sum\limits_{k}^{\rm lens} \tilde{w}^\prime_k \Sigma_{c,k}^{-1}\Delta\Sigma}{\sum\limits_{k}^{\rm lens} \tilde{w}^\prime_k}+\frac{B^\prime(r_p) \sum\limits_{k}^{\rm lens} \tilde{w}^\prime_k a \gamma_{\rm IA}^k}{\sum\limits_{k}^{\rm lens} \tilde{w}^\prime_k} 
\label{full_both}
\end{align}
Note that the weights and anything that depends on the weights are different between the two shape-measurement methods because of the different value of $e_{\rm rms}$ between the two methods. I don't know exactly what the weights should be in the case of averaged tangential shear as the observable, but I assume they depend on $e_{\rm rms}$.

Subtract as in equation \ref{gen_2pt} to get:
\begin{align}
\tilde{\gamma}(r_p) - \tilde{\gamma}^\prime(r_p) &= \Delta \Sigma(r_p)\left[\frac{B(r_p) \sum\limits_{k}^{\rm lens} \tilde{w}_k \Sigma_{c,k}^{-1}}{\sum\limits_{k}^{\rm lens} \tilde{w}_k} - \frac{B^\prime(r_p) \sum\limits_{k}^{\rm lens} \tilde{w}^\prime_k \Sigma_{c,k}^{-1}}{\sum\limits_{k}^{\rm lens} \tilde{w}^\prime_k}\right] \nonumber \\ &+\gamma_{\rm IA}(r_p) \left[\frac{B(r_p) \left(\sum\limits_{k}^{\rm excess} \tilde{w}_k + \sum\limits_{k}^{\rm rand-close} \tilde{w}_k\right)}{\sum\limits_{k}^{\rm lens} \tilde{w}_k}-a\frac{B^\prime(r_p) \left(\sum\limits_{k}^{\rm excess} \tilde{w}^\prime_k  + \sum\limits_{k}^{\rm rand-close} \tilde{w}^\prime_k\right)}{\sum\limits_{k}^{\rm lens} \tilde{w}^\prime_k} \right]
\label{subtract_gammat}
\end{align}

We have made the assumption that we can pull an average per-contributing-galaxy value of $\gamma_{IA}(r_p)$ out of the sums, and that excess galaxies as well as nearby galaxies from the smooth source sample (`rand-close') can contribute to intrinsic alignments. Using the same math as in the Blazek et al. 2012 case, we can equally write this as:  
\begin{align}
\tilde{\gamma}(r_p) - \tilde{\gamma}^\prime(r_p) &= \Delta \Sigma(r_p)\left[\frac{B(r_p) \sum\limits_{k}^{\rm lens} \tilde{w}_k \Sigma_{c,k}^{-1}}{\sum\limits_{k}^{\rm lens} \tilde{w}_k} - \frac{B^\prime(r_p) \sum\limits_{k}^{\rm lens} \tilde{w}^\prime_k \Sigma_{c,k}^{-1}}{\sum\limits_{k}^{\rm lens} \tilde{w}^\prime_k}\right] +\gamma_{\rm IA}(r_p) \left[(B(r_p)-1+F)-a(B^\prime(r_p)-1+F^\prime) \right]
\label{subtract_gammat_2}
\end{align}
where we have used the fact that:
\begin{equation}
\sum\limits_{k}^{\rm lens} \tilde{w}_k = \frac{B(r_p)}{B(r_p)-1+F} \left(\sum\limits_{k}^{\rm excess} \tilde{w}_k + \sum\limits_{k}^{\rm rand-close} \tilde{w}_k\right)
\label{lens_weights_expand}
\end{equation}
and we have defined:
\begin{equation}
F = \frac{\sum\limits^{\rm rand-close}_{k} \tilde{w}_k}{\sum\limits^{\rm rand(all)}_{k} \tilde{w}_k}.
\label{Frp}
\end{equation}

If it is true that:
\begin{equation}
\frac{B(r_p) \sum\limits_{k}^{\rm lens} \tilde{w}_k \Sigma_{c,k}^{-1}}{\sum\limits_{k}^{\rm lens} \tilde{w}_k} \approx \frac{B^\prime(r_p) \sum\limits_{k}^{\rm lens} \tilde{w}^\prime_k \Sigma_{c,k}^{-1}}{\sum\limits_{k}^{\rm lens} \tilde{w}^\prime_k}
\label{lens_req}
\end{equation}
and
\begin{equation}
(B(r_p)-1+F) \approx (B^\prime(r_p)-1+F^\prime)
\label{ia_req}
\end{equation}
then we can simplify equation \ref{subtract_gammat_2} and solve for the IA contribution. {\it Is this true?} In this case, we have:
\begin{equation}
\gamma_{IA}(r_p)(1-a) = \frac{\tilde{\gamma}(r_p) - \tilde{\gamma}'(r_p)}{(B(r_p)-1+F)}
\label{gammaIA_sol}
\end{equation}

The above result demonstrates that as in the case of the Blazek 2012 formalism, we need to worry about dividing quantities with errors. We expect the error on the numerator and denominator to be uncorrelated (the numerator is dominated by shape-noise while the error on the denominator is sourced from large-scale structure), but we still need to be concerned about the fact that should we need the covariance in $r_p$ bins, this is a more non-trivial thing to compute. (The variance can be obtained from the basic propagation of errors formula).

To estimate the statistical error on the left hand side of equation \ref{gammaIA_sol}, we can use the expression:
\begin{equation}
\frac{\sigma^2(\gamma^{\rm IA}(r_p)(1-a))}{(\gamma^{\rm IA}(r_p)(1-a))^2} = \frac{\sigma^2(\tilde{\gamma}(r_p) - \tilde{\gamma}(r_p)^\prime)}{(\tilde{\gamma}(r_p) - \tilde{\gamma}(r_p)^\prime)^2} + \frac{\sigma^2(B(r_p)-1+F)}{(B(r_p)-1+F)^2}
\label{error_gt}
\end{equation}

\subsection*{Caveats to the above}
\begin{itemize}
\item{I have assumed that the weights in this case (where we are calculating the average tangential shear instead of $\Delta \Sigma$) are independent of photometric redshift. I'm not sure if this is right. If the weights depend on photometric redshift, we need to compute a photometric redshift bias similar to the Blazek et al. 2012 case.}
\end{itemize}

\section*{Observable: $\Delta \Sigma$}

Instead of considering the averaged tangential shear as the two-point function in equation \ref{gen_2pt}, we could use $\Delta \Sigma$. It seems from a brief lit review that people have been using $\Delta \Sigma$ pretty exclusively for lensing since having access to any kind of redshift information, so maybe we should be using this here as well.

Using $\Delta \Sigma$ for the observable, equation \ref{subtract_gammat} becomes
\begin{align}
\tilde{\Delta \Sigma}(r_p) &= \frac{B(r_p) \sum\limits_{k}^{\rm lens} \tilde{w}_k \tilde{\Sigma}_c^k\tilde{\gamma}_k}{\sum\limits_{k}^{\rm lens} \tilde{w}_k} = \frac{B(r_p) \sum\limits_{k}^{\rm lens} \tilde{w}_k \tilde{\Sigma}_c^k \Sigma_{c,k}^{-1}\Delta\Sigma}{\sum\limits_{k}^{\rm lens} \tilde{w}_k}+\frac{B(r_p) \sum\limits_{k}^{\rm lens} \tilde{w}_k \tilde{\Sigma}_c^k \gamma_{\rm IA}^k}{\sum\limits_{k}^{\rm lens} \tilde{w}_k} \nonumber \\ 
\tilde{\Delta\Sigma}^\prime(r_p) &= \frac{B^\prime(r_p) \sum\limits_{k}^{\rm lens} \tilde{w}^\prime_k \tilde{\Sigma}_c^k\tilde{\gamma}_k}{\sum\limits_{k}^{\rm lens} \tilde{w}^\prime_k} = \frac{B^\prime(r_p) \sum\limits_{k}^{\rm lens} \tilde{w}^\prime_k \tilde{\Sigma}_c^k \Sigma_{c,k}^{-1}\Delta\Sigma}{\sum\limits_{k}^{\rm lens} \tilde{w}^\prime_k}+\frac{B^\prime(r_p) \sum\limits_{k}^{\rm lens} \tilde{w}^\prime_k a \tilde{\Sigma}_c^k\gamma_{\rm IA}^k}{\sum\limits_{k}^{\rm lens} \tilde{w}^\prime_k} 
\label{full_both_Delta_Sigma}
\end{align}
and equation \ref{subtract_gammat_2} becomes:
\begin{align}
\tilde{\Delta \Sigma}(r_p) - \tilde{\Delta\Sigma}^\prime(r_p) &= \Delta \Sigma(r_p)\left[\frac{B(r_p) \sum\limits_{k}^{\rm lens} \tilde{w}_k \tilde{\Sigma}_c^k\Sigma_{c,k}^{-1}}{\sum\limits_{k}^{\rm lens} \tilde{w}_k} - \frac{B^\prime(r_p) \sum\limits_{k}^{\rm lens} \tilde{w}^\prime_k \tilde{\Sigma}_c^k\Sigma_{c,k}^{-1}}{\sum\limits_{k}^{\rm lens} \tilde{w}^\prime_k}\right] \nonumber \\ &+\gamma_{\rm IA}(r_p) \left[(B(r_p)-1+F)\langle \Sigma_c \rangle_{\rm IA}-a(B^\prime(r_p)-1+F^\prime)\langle \Sigma_c \rangle_{\rm IA}^\prime \right]
\label{subtract_deltaS}
\end{align}

In this case, in order to simplify the above to solve for $\gamma_{\rm IA}$, we would need:
\begin{equation}
\frac{B(r_p) \sum\limits_{k}^{\rm lens} \tilde{w}_k \tilde{\Sigma}_c^k\Sigma_{c,k}^{-1}}{\sum\limits_{k}^{\rm lens} \tilde{w}_k} \approx \frac{B^\prime(r_p) \sum\limits_{k}^{\rm lens} \tilde{w}^\prime_k \tilde{\Sigma}_c^k\Sigma_{c,k}^{-1}}{\sum\limits_{k}^{\rm lens} \tilde{w}^\prime_k}
\label{lens_req_DS}
\end{equation}
and
\begin{equation}
(B(r_p)-1+F)\langle \Sigma_c^{\rm IA} \rangle\approx (B^\prime(r_p)-1+F^\prime)\langle \Sigma_c ^{\rm IA}\rangle^\prime
\label{ia_req_DS}
\end{equation}

In that case, we then find:
\begin{equation}
\gamma_{IA}(r_p)(1-a) = \frac{\tilde{\Delta \Sigma}(r_p) - \tilde{\Delta \Sigma}'(r_p)}{\langle \Sigma_c^{\rm IA}\rangle (B(r_p)-1+F)}
\label{gammaIA_sol_DS}
\end{equation}
where $\langle \Sigma_c^{\rm IA}\rangle$ is defined as in the Blazek et al. 2012 case.

To estimate the statistical errors on the left hand side in this case, the required expression is:
\begin{equation}
\frac{\sigma^2(\gamma^{\rm IA}(r_p)(1-a))}{(\gamma^{\rm IA}(r_p)(1-a))^2} = \frac{\sigma^2(\tilde{\Delta \Sigma}(r_p) - \tilde{\Delta \Sigma}(r_p)^\prime)}{(\tilde{\Delta \Sigma}(r_p) - \tilde{\Delta \Sigma}(r_p)^\prime)^2} + \frac{\sigma^2(B(r_p)-1+F)}{(B(r_p)-1+F)^2}
\label{error_DS}
\end{equation}
Note that unlike in the case of the Blazek et al. 2012 formalism, $\langle \Sigma_c^{\rm IA}\rangle$ has canceled out of the expression for statistical error. 

As in the case of using tangential shear as the observable, remember to account for correlation between the two shape noises and to somehow select an $a$ value for computing the denominator of the first term on the right hand side.

\section*{Questions}
\begin{enumerate}
\item{Should we be using $\Delta \Sigma$ or averaged tangential shear as the observable?}
\item{If $\gamma_t$, what should be done about the weights? Are they the same as in the $\Delta \Sigma$ case?}
\item{Should we be using $\Delta\Sigma$ as the `theory' quantity?}
\item{The fact that $e_{\rm rms}$ is different for different shape measurements means that anything that depends on the weights is different too for each measurement, as shown. This needs to be tested. I could test it directly, but I'm not sure how to treat e.g. the boost which is currently taken as an empirical model.}
\end{enumerate}
%-------------------------------------------------------------------------------


%-------------------------------------------------------------------------------

\end{document}
