\documentclass[onecolumn,amsmath,aps,fleqn, superscriptaddress]{revtex4}
 
\usepackage{amssymb}
\usepackage{amsmath}
\usepackage{epsfig}
\usepackage{subfigure}
\usepackage{mathrsfs}
\usepackage{longtable}
\usepackage{enumerate} 
\usepackage{multirow}
\usepackage{color}

\usepackage[usenames,dvipsnames]{xcolor}
\usepackage{hyperref}
\hypersetup{
    colorlinks = true,
    citecolor = {MidnightBlue},
    linkcolor = {BrickRed},
    urlcolor = {BrickRed}
}

\newcommand{\be}{\begin{eqnarray}}
\newcommand{\ee}{\end{eqnarray}}

\allowdisplaybreaks[1]
 
\begin{document}

\title{Intrinsic alignments from different shape measurement techniques}

\author{Danielle Leonard}

\maketitle

\paragraph*{Background} Different shape measurement methods are sensitive to different parts of the radial profile of a galaxy. In Singh et al. 2016, it appears as though the intrinsic alignment signals measured by these different methods are offset from one another by a constant. By measuring a correlation between galaxy position and shape with two different shape measurements, and subtracting these, you could get rid of the lensing signal and obtain a constraint on the scale-dependence of intrinsic alignments. This is useful for building intrinsic alignment models.

\paragraph*{Source sample} This method naturally allows us to loosen the standard assumption that only excess galaxies are subject to IA. We consider galaxies which form part of the smooth source distribution, but which are still physically close to a lens galaxy, to be impacted by intrinsic alignments as well. We assume a source sample which is defined by the requirement that the photo-z of a source galaxy is always within a certain cut-off separation of the lens galaxy in question. In the case of an effective redshift for the lenses, call the high and low photo-z edges of the sample $z_{\rm cut_h}$ and $z_{\rm cut_l}$, respectively. Note that galaxies with photo-z below the effective lens redshift are included in the sample, because they are subject to intrinsic alignments. 

We write:
\begin{align}
\tilde{\gamma}(r_p) &= \frac{\sum\limits_{k}^{\rm lens} \tilde{w}_k \tilde{\gamma}_k}{\sum\limits_{k}^{\rm lens} \tilde{w}_k} = \frac{\sum\limits_{k}^{\rm lens} \tilde{w}_k \gamma_{\rm L}^k}{\sum\limits_{k}^{\rm lens} \tilde{w}_k}+\frac{\sum\limits_{k}^{\rm lens} \tilde{w}_k \gamma_{\rm IA}^k}{\sum\limits_{k}^{\rm lens} \tilde{w}_k} \nonumber \\ 
\tilde{\gamma}'(r_p) &= \frac{\sum\limits_{k}^{\rm lens} \tilde{w}_k \tilde{\gamma}_k}{\sum\limits_{k}^{\rm lens} \tilde{w}_k} = \frac{\sum\limits_{k}^{\rm lens} \tilde{w}_k \gamma_{\rm L}^k}{\sum\limits_{k}^{\rm lens} \tilde{w}_k}+\frac{\sum\limits_{k}^{\rm lens} \tilde{w}_k a \gamma_{\rm IA}^k}{\sum\limits_{k}^{\rm lens} \tilde{w}_k} 
\label{full_both}
\end{align}
where all shear should be assumed to be tangential about the lens galaxy. When computing averaged tangential shear about a lens galaxy, the inverse-variance weights are:
\begin{equation}
\tilde{w}_k = \frac{1}{e_{\rm rms}^2 + (\sigma_e^k)^2}
\label{weights}
\end{equation}

A couple of notes on equation \ref{full_both}:
\begin{enumerate}
\item{The optimal choice for the two-point function here is the averaged tangential shear rather than $\Delta \Sigma$. $\Delta \Sigma$ is ill-defined for source galaxies which are very near in redshift to the lens galaxy or in front of the lens galaxy. By definition $\Delta \Sigma$ will be ill-defined for much of our sample.}
\item{The weights in equation \ref{full_both} are the same for both shape measurement methods. This is not an obvious thing to choose, because $e_{\rm rms}$ would be expected to vary between the methods. But, it's important to choose the weights the same so that lensing will cancel when we do the subtraction. Also, it turns out that because $e_{\rm rms}$ dominates over measurement error, choosing a non-optimal weight by changing $e_{\rm rms}$ has little effect on how optimal the weights are, because we are basically weighting every pair the same regardless. So, for the weights, we pick some $e_{\rm rms}$ which is between the values for the two shape measurement methods.}
\item{The boost factor is not included in equation \ref{full_both}. This is because the point of the boost factor is that if you are trying to get a measurement of lensing by summing tangential shears, then you need the boost factor or else your measured signal will be low. But here, we are not trying to get the lensing signal. We are interested in the intrinsic alignment signal, and the lensing signal will cancel with or without the boost, so it is not necessary.}
\end{enumerate}

We subtract the above relationships for the two shape measurement methods to get:
\begin{align}
\tilde{\gamma}(r_p) - \tilde{\gamma}^\prime(r_p) &= \frac{\sum\limits_{k}^{\rm lens} \tilde{w}_k \gamma_{\rm IA}^k}{\sum\limits_{k}^{\rm lens} \tilde{w}_k} - \frac{\sum\limits_{k}^{\rm lens} \tilde{w}_k a \gamma_{\rm IA}^k}{\sum\limits_{k}^{\rm lens} \tilde{w}_k} 
\label{subtract_gammat}
\end{align}
In principle, all the galaxies in our source sample contribute to intrinsic alignments, so we can consider an average per-galaxy contribution of $\gamma_{\rm IA}(r_p)$.
\begin{equation}
\tilde{\gamma}(r_p) - \tilde{\gamma}^\prime(r_p) = \gamma_{\rm IA}(r_p)(1-a).
\label{subtract_gammat_simp}
\end{equation}
However, there is another factor here that we have not yet accounted for. Because we are assuming photometric redshifts for our sources, we will have source galaxies in our sample which are actually higher redshift than $z_{\rm cut_h}$ and at lower redshift than $z_{\rm cut_l}$. These galaxies will not be affected by intrinsic alignments. Therefore, if we apply equation \ref{subtract_gammat_simp} directly, we will estimate the intrinsic alignment signal to be too low. We need to apply a correction factor which accounts for this. 

\subsection*{Correction factor: inclusion of higher-z and lower-z galaxies}
We assume that we know the smooth distribution of galaxies, $dN / dz$, from which our source sample is drawn to some reasonably high redshift, from a spectroscopic subsample. This may be difficult for future surveys, in which an inadequate spectroscopic subsample could result in systematic error (discussed below). 

Let $z_{\rm min}$ and $z_{\rm max}$ be, respectively, the minimum and maximum spec-z, and let the probability of labelling a galaxy with spectroscopic redshift $z_s$ as having photometric redshift $z_p$ be $p(z_s, z_p)$. The required correction factor is the fraction of galaxies in the photo-z-defined sample which have spec-z between $z_{\rm cut_l}$ and $z_{\rm cut_h}$. This can be separated into two terms:
\begin{enumerate}
\item{The galaxies from the smooth redshift distribution which have spec-z between $z_{\rm cut_l}$ and $z_{\rm cut_h}$.}
\item{The excess galaxies. All of these have spec-z between $z_{\rm cut_l}$ and $z_{\rm cut_h}$ by definition.}
\end{enumerate}
Schematically, we have:
\begin{equation}
N_{\rm corr} = \frac{{\rm smooth, z_s \, in \, (z_{\rm cut_l}, z_{\rm cut_h})} + {\rm excess}}{{\rm total}} =  \frac{{\rm smooth, z_s \, in \, (z_{\rm cut_l}, z_{\rm cut_h})}}{{\rm total}} + \frac{{\rm excess}}{{\rm total}}
\label{schematic}
\end{equation}
where these quantities should all be considered as weighted. 

The weighted fraction of galaxies which are excess, i.e., the second term above, is given by:
\begin{equation}
1 - \frac{1}{B(r_p)}.
\label{frac_excess}
\end{equation}

The numerator of the first term on the right hand side of equation \ref{schematic}, that is, the weighted number of galaxies from the smooth distribution which have spec-z within the photo-z cuts, is given by:
\begin{equation}
\int_{z_{\rm cut_l}}^{z_{\rm cut_h}} dz_p \tilde{w}(z_p) \int_{z_{\rm cut_l}}^{z_{\rm cut_h}} dz_s\frac{dN}{dz_s} p(z_s, z_p)
\label{num_smooth_speczin}
\end{equation}

The total weighted number of galaxies from the {\it smooth} distribution can be given similarly by:
\begin{equation}
\int_{z_{\rm cut_l}}^{z_{\rm cut_h}} dz_p  \tilde{w}(z_p) \int_{z_{\rm min}}^{z_{\rm max}} dz_s\frac{dN}{dz_s} p(z_s, z_p)
\label{tot_smooth}
\end{equation}
However, this is not the denominator of the first term on the right hand side of equation \ref{schematic}, for this we need the total weighted number of galaxies, not the total only from the smooth distribution. Fortunate, the boost is defined as (schematically):
\begin{equation}
B(r_p) = \frac{ {\rm total}}{{\rm total, smooth}}
\label{boost_schm}
\end{equation}
It can therefore be seen that multiplying equation \ref{tot_smooth} by the boost gives the desired total. Assembling these terms, we find the correction factor to be given by:
\begin{equation}
N_{\rm corr}(r_p) = \frac{\int_{z_{\rm cut_l}}^{z_{\rm cut_h}} dz_p \tilde{w}(z_p) \int_{z_{\rm cut_l}}^{z_{\rm cut_h}} dz_s\frac{dN}{dz_s} p(z_s, z_p)}{B(r_p) \int_{z_{\rm cut_l}}^{z_{\rm cut_h}} dz_p  \tilde{w}(z_p) \int_{z_{\rm min}}^{z_{\rm max}} dz_s\frac{dN}{dz_s} p(z_s, z_p)} +1 - \frac{1}{B(r_p)}.
\label{Ncorr}
\end{equation}
If desired, this can be rearranged into the less pedagogical but potentially more useful form:
\begin{equation}
N_{\rm corr}(r_p) = 1 - \frac{1}{B(r_p)}\left(1 - \frac{\int_{z_{\rm cut_l}}^{z_{\rm cut_h}} dz_p \tilde{w}(z_p) \int_{z_{\rm cut_l}}^{z_{\rm cut_h}} dz_s\frac{dN}{dz_s} p(z_s, z_p)}{\int_{z_{\rm cut_l}}^{z_{\rm cut_h}} dz_p  \tilde{w}(z_p) \int_{z_{\rm min}}^{z_{\rm max}} dz_s\frac{dN}{dz_s} p(z_s, z_p)} \right)
\label{ncorr_2}
\end{equation}

We alter equation \ref{subtract_gammat_simp} to account for this correction factor
\begin{equation}
\tilde{\gamma}(r_p) - \tilde{\gamma}^\prime(r_p) = \gamma_{\rm IA}(r_p)(1-a) N_{\rm corr}(r_p)
\label{subtract_gammat_fac}
\end{equation}
which is trivially rearranged to solve for the IA term:
\begin{equation}
\gamma_{\rm IA}(r_p)(1-a) = \left(\tilde{\gamma}(r_p) - \tilde{\gamma}^\prime(r_p)\right)\left(\frac{1}{N_{\rm corr}(r_p)}\right)
\label{IAsol}
\end{equation}

\subsection*{Statistical error}

There are two types of statistical error sources in equation \ref{IAsol}: 
\begin{enumerate}
\item{$\tilde{\gamma}(r_p)$ and $\tilde{\gamma}^\prime(r_p)$ have statistical error due to shape noise}
\item{$N_{\rm corr}(r_p)$ has statistical error due to $B(r_p)$, which is sourced from large scale structure}
\end{enumerate}
It is expected that the second of these will be sub-dominant, but we include it anyway for completeness. 

These two sources of error are independent, so the total statistical error contribution can be written as:
\begin{align}
\sigma^2(\gamma_{\rm IA}(r_p)(1-a)) &= (1-a_{\rm fid})^2\gamma_{\rm IA}^{\rm fid}(r_p)^2\left(\frac{\sigma^2(N_{\rm corr}(r_p))}{N_{\rm corr}^{\rm fid}(r_p)^2} + \frac{\sigma^2(\tilde{\gamma}(r_p) - \tilde{\gamma}^\prime(r_p))}{(\tilde{\gamma}(r_p) - \tilde{\gamma}^\prime(r_p))^2}\right) \nonumber \\
\sigma^2(\gamma_{\rm IA}(r_p)(1-a)) &= (1-a_{\rm fid})^2\gamma_{\rm IA}^{\rm fid}(r_p)^2\left(\frac{\sigma^2(N_{\rm corr}(r_p))}{N_{\rm corr}^{\rm fid}(r_p)^2} + \frac{\sigma^2(\tilde{\gamma}(r_p) - \tilde{\gamma}^\prime(r_p))}{(N_{\rm corr}^{\rm fid}(1-a_{\rm fid})^2\gamma_{\rm IA}^{\rm fid}(r_p))^2}\right)
\label{error}
\end{align}
Note that in the above equation we explicitly consider only diagonal elements of the covariance matrix between $r_p$ bins. {\it The statistical error contribution from $N_{\rm corr}$ may have off-diagonal elements. For the moment we don't consider these.}

$\sigma^2(N_{\rm corr}(r_p))$ is given by:
\begin{equation}
\sigma^2(N_{\rm corr}(r_p)) = \frac{\sigma^2(B(r_p))}{B(r_p)^4}\left[1-\frac{\int_{z_{\rm cut_l}}^{z_{\rm cut_h}} dz_p \tilde{w}(z_p) \int_{z_{\rm cut_l}}^{z_{\rm cut_h}} dz_s\frac{dN}{dz_s} p(z_s, z_p)}{\int_{z_{\rm cut_l}}^{z_{\rm cut_h}} dz_p  \tilde{w}(z_p) \int_{z_{\rm min}}^{z_{\rm max}} dz_s\frac{dN}{dz_s} p(z_s, z_p)}\right]
\label{Ncorr_err}
\end{equation}

The error on the averaged tangential shear obtained using each shape-measurement method is dominated by shape noise, so the covariance can be represented by a diagonal matrix with 
\begin{equation}
\frac{e_{\rm rms}^2}{N_{ls}}
\label{diag_corr}
\end{equation} 
on the diagonal for each bin and the correct choice of $e_{\rm rms}$ for each shape measurement (as opposed to the average $e_{\rm rms}$ which is used in computing the weights). The errors for the two shape measurements are correlated so this must be taken into account when combining them. 

When computing the expression of equation \ref{diag_corr}, one has to compute $N_{ls}$ in each projected radial bin. Given that we have a surface density of galaxies for the full $dN/dz$, the fraction of these which lie within the photo-z cuts of our source sample should be considered. When computing this fraction, the $dN / dz$ with respect to photo-z's should be used. {\it It is also important to include the excess galaxies when calculating $N_{ls}$. I currently haven't done this.}

\subsection*{Systematic error}

There are similarly two obvious sources of systematic error in equation \ref{IAsol}:
\begin{enumerate}
\item{$\frac{dN}{dz_s}$: We may have an imperfect knowledge of $\frac{dN}{dz_s}$ due to an inadequate spectroscopic subsample, especially for dim source galaxies.}
\item{$p(z_s, z_p)$: We will not necessarily have an accurate model of how likely we are to measure $z_p$ given $z_s$.}
\end{enumerate}

Uncertainty in either $\frac{dN}{dz_s}$ or $p(z_s, z_l)$ leads to uncertainty in the $\frac{1}{N_{\rm corr}}$ term of equation \ref{IAsol}. This manifests as a systematic error here because it affects the measurements of the IA term in each projected $r_p$ bin in the same way. For example, if $\frac{1}{N_{\rm corr}}$ is high, then $(1-a)\gamma_{\rm IA}(r_p)$ would be high in each bin. This introduces a correlation between $r_p$ bins as well as a contribution to the diagonal elements to the covariance. {\it For the moment, we are only considering the diagonal elements of the covariance.}

The systematic error contributions from these sources are given by:
\begin{align}
S_{dN}^2 &= \sigma_{dN}^2 (1-a_{\rm fid})^2 N_{\rm corr}^{{\rm fid}}(r_p) N_{\rm corr}^{{\rm fid}}(r_p')\gamma_{\rm IA}^{\rm fid}(r_p) \gamma_{\rm IA}^{\rm fid}(r_p') \nonumber \\
S_{p}^2 &= \sigma_{p}^2 (1-a_{\rm fid})^2 N_{\rm corr}^{{\rm fid}}(r_p) N_{\rm corr}^{{\rm fid}}(r_p')\gamma_{\rm IA}^{\rm fid}(r_p) \gamma_{\rm IA}^{\rm fid}(r_p') 
\label{S2}
\end{align}
where `fid' indicates a fiducial value.

In order to get an estimate for $\sigma_{\rm dN}$, we would vary $\frac{dN}{dz}$. This could be done, for example, by varying the parameters of $\frac{dN}{dz}$ over the range of their uncertainty (given in Nakajima 2011 for SDSS). Then, for each of these choices of $\frac{dN}{dz}$, we would compute $\frac{1}{N_{\rm corr}}$. The range of resultant $N_{\rm corr}$ would allow for an estimate of $\sigma_{\rm dN}$. A similar scheme could be used to account for $\sigma_{\rm p}$.

\subsection*{Total covariance matrix}

The diagonal elements of the total covariance matrix are:
\begin{equation}
C(r_p,r_p) = \sigma^2(r_p) + S_{dN}^2(r_p, r_p) + S_{p}^2(r_p, r_p)
\label{diag}
\end{equation}
and the off-diagonal elements are given by
\begin{equation}
C(r_p, r_p') = S_{dN}^2(r_p,r_p')+S_{p}^2(r_p,r_p')
\label{off_diag}
\end{equation}
{\it under the likely-incorrect assumption that the statistical error from the boost has no off-diagonal elements}.

\subsection*{Factors which affect errors}
There are numerous degrees of freedom in the calculation which can affect the calculated errors on $(1-a)\gamma_{\rm IA}$. We break them up into several categories.

Directly related to the covariance matrix:
\begin{itemize}
\item{$e_{\rm rms}$ for each shape measurement method}
\item{the density of lens galaxies}
\item{the density of source galaxies}
\item{covariance of $\gamma_t$ between the two shape-measurement methods}
\item{the level of systematic error on $\frac{1}{N_{\rm corr}}$}
\item{fiducial model / values for $\gamma_{\rm IA}$, $a$, and $\frac{dN}{dz_s}$}
\item{the photo-z model}
\end{itemize}

Related to redshift setup (and so indirectly related to the covariance matrix):
\begin{itemize}
\item{the redshift of the lenses}
\item{the line-of-sight separation from the lens within which we assume galaxies are subject to IA}
\item{Cosmological parameters (affect $z$ to $\chi$ conversion)}
\end{itemize}

Currently no effect but could have an effect if weights were redshift-dependent:
\begin{itemize}
\item{the signal to noise (affects $\sigma_e$)}
\item{the intermediate $e_{\rm rms}$ which goes into the weights}
\end{itemize}

\section*{Questions}
\begin{enumerate}
\item{When computing the number of galaxy pairs to use in computing $N_{ls}$ for the covariance, how to include excess galaxies?}
\item{How to deal with off-diagonal elements of the contribution to the statistical error from the boost?}
\end{enumerate}
%-------------------------------------------------------------------------------


%-------------------------------------------------------------------------------

\end{document}
