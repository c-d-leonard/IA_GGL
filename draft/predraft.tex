\documentclass[twocolumn,amsmath,aps,fleqn, superscriptaddress]{revtex4}
 
\usepackage{amssymb}
\usepackage{amsmath}
\usepackage{epsfig}
\usepackage{subfigure}
\usepackage{mathrsfs}
\usepackage{longtable}
\usepackage{enumerate} 
\usepackage{multirow}
\usepackage{color}

\usepackage[usenames,dvipsnames]{xcolor}
\usepackage{hyperref}
\hypersetup{
    colorlinks = true,
    citecolor = {MidnightBlue},
    linkcolor = {BrickRed},
    urlcolor = {BrickRed}
}

\newcommand{\be}{\begin{eqnarray}}
\newcommand{\ee}{\end{eqnarray}}

\newcommand{\corr}{\color{blue}}

\newcommand{\mnras}{MNRAS~}
\newcommand{\jcap}{JCAP~}
%\newcommand{\apj}{ApJ~}
\newcommand{\apjl}{ApJL~}
\newcommand{\apjs}{ApJS~}

 \allowdisplaybreaks[1]
 
\begin{document}

\title{{Constraints on Intrinsic Alignments from Multiple Shape-Measurement Methods}}

\author{C. Danielle Leonard}
\email{danielll@andrew.cmu.edu}
\affiliation{McWilliams Center for Cosmology, Department of Physics, Carnegie Mellon University, 5000 Forbes Avenue, Pittsburgh, PA 15213, USA}

\author{Rachel Mandelbaum}
\affiliation{McWilliams Center for Cosmology, Department of Physics, Carnegie Mellon University, 5000 Forbes Avenue, Pittsburgh, PA 15213, USA}

\date{\today}

\begin{abstract}
This is a pre-draft of a project about constraining the scale-dependence of the intrinsic-alignment signal using multiple shape-measurement methods.
\end{abstract}


\maketitle


%-------------------------------------------------------------------------------
\section{Introduction}
\label{sec:introduction}
\noindent
Weak lensing is an important way to constrain cosmology.

Intrinsic alignments are an important contaminant to weak lensing measurements. We need to be able to constrain their contribution well to believe our weak lensing cosmological constraints.

This project looks at a couple of improvements to existing methods of constraining intrinsic alignemnts. One is to use the existing method of \cite{Blazek2012}, but instead of assuming that IA only affect excess galaxies, include IA effects on physically close galaxies which are not excess. The second is to take advantage of the recent finding that intrinsic alignment signals as obtained from different shape-measurement signals result in IA signal which are offset by a constant value \cite{Singh2016}. 

%-------------------------------------------------------------------------------
\section{Background}
\label{sec:theory}
\subsection{Weak Lensing Theory}
\label{subsec:wltheory}
Here is some information about how we assume things are measured for galaxy-galaxy lensing and how intrinsic alignments come in to weak lensing.

\subsection{Existing methods for constraining intrinsic alignments}
\label{subsec:existingIA}
Here are the equations for Jonathan's method as they are are given in his paper.

\section{Including physically associated non-excess galaxies}
\label{sec:closerand}
Here are the equations for including F in Jonathan's method.

\section{Constaining IA with multiple shape-measurement methods}
\label{subsec:newmethod}
Here is the theory / set up for our method of constraining IA.


\section{Results}
\label{sec:results}
\noindent
\subsection{Including non-excess galaxies in the Blazek et al. 2012 method}
\label{subsec:results_F}
\noindent
Modifying the Blazek et al. 2012 method to include non-excess physically associated galaxies significantly reduces the resulting error on large scales (see Figure \ref{fig:Fzerocomparison}).

\begin{figure*}
\centering
\subfigure{\includegraphics[width=0.45\textwidth]{stat+sys_notloBlazekMethod_LRG-shapes_7bins.pdf}}
\subfigure{\includegraphics[width=0.45\textwidth]{stat+sys_notloBlazekMethod_LRG-shapes_7bins_F=0.pdf}}
\caption{Left: Statistical + systematic error for the modified Blazek et al. method, including non-excess physically associated galaxies. Right: Same, but for the original Blazek et al. method, assuming IA affect only excess galaxies.}
\label{fig:Fzerocomparison}
\end{figure*}

\subsection{Systematic error due to photometric redshift issues}
\label{subsec:results_sys}
\noindent
The expression for $\gamma_{\rm IA}$ within the modified method of \cite{Blazek2012} (equation something) depends on several quantities which are sensitive to photometric redshift calibration ($c_z$, $\langle \Sigma_{\rm IA} \rangle$, and $F$ for each sample). On the other hand, within the proposed method, $\gamma_{\rm IA}$ depends on photometric redshift calibration only via $N_{\rm corr}$ (see equation something). We therefore investigate whether the proposed method may result in reduced systematic uncertainty in the case in which our spectroscopic subsample of source galaxies in inadequately representative . 

We first investigate the relative importance of systematic uncertainty on each of the quantities in equation (Blazek expression, something) which are senstive to photometric redshift calibration. We do so by introducing a fractional systematic error on each of these quantities (in each case holding the equivalent fractional systematic error on the remaining quantities fixed at zero). We do so for each quantity at fractional error levels ranging from 2\% to 100\%. Figure \ref{fig:StoNsysvstat} shows the result of this exercise. The quantity plotted is the ratio of signal-to-noise assuming systematic error only vs signal-to-noise assuming statistical error only, as a function of assumed fractional systematic error on the given quantity. To be clear, a value of unity indicates that statistica and systematic errors are of equal importance; the greater the value, the more the statistical error dominates.

We see from Figure \ref{fig:StoNsysvstat} that of the three types of terms affected by photo-z-related systematic error, the method is most sensitive to resulting uncertainty in $c_z$, the photo-z bias terms. In order to ensure that statistical errors dominate over systematic errors, we have to control, for example, $c_z^a$ to better than about $6\%$. 

The fact that it is $c_z$ to which the method is most sensitive of these three quantities has implications for the relative utility of the Blazek et al. method vs the proposed method. $c_z$ is the photometric redshift bias to $\tilde{\Delta \Sigma}$. If systematic uncertainty to this quantity is poorly controlled, then it is in fact not possible to measure $\tilde{\Delta \Sigma}$ for the purposes of lensing, rendering the question of which method better mitigates the intrinsic alignment contribution to the signal moot. Therefore, we must assuming that systematic error on $c_z$ is well-controlled for any scenario we care about. The implication is that, in order for systematic error to dominate over statistical error for the Blazek et al. method (and therefore for it to be an interesting question of whether the proposed method may help mitigate this sytematic error), there must be a scenario in which the fractional systematic uncertainty on either $\langle \Sigma_{\rm IA} \rangle$ or $F$ is larger than that of $c_z$. However, recall that $\langle \Sigma_{\rm IA} \rangle$ and $F$ are sensitive to source galaxies which are subject to intrinsic alignment by definition - that is, they are at or around the redshift of the lens distribution. $c_z$, on the other hand, is sensitive to all source galaxies in either the `a' or 'b' sample. It is unlikely that our photo-z calibration would ever be worse near the lens distribution that at higher redshifts, especially where we assume we use a lens sample with spectroscopic redshifts. Therefore, we conclude that there is unlikely to be any scenario of relevance in which the proposed method can improve upon the method of \cite{Blazek2012} due to its reduced dependence on systematic errors related to photometric redshifts.

\begin{figure*}
\centering
\includegraphics[width=0.75\textwidth]{ratio_StoN.pdf}
\caption{S / N for the hypothetical case in which the only source of error comes from systematic uncertainty to the various quantities listed, divided by S / N for the hypothetical case with only statistical error. For statistical error to dominate, the plotted quantity should be unity or greater.}
\label{fig:StoNsysvstat}
\end{figure*}

\subsection{Statistical error only}
\label{subsec:results_stat}
\noindent
As described in the previous section, the requirement that the photo-z bias to the lensing signal be well-known effectively forces the scenario in which statistical errors dominate over systematic errors related to photometric redshift effects. Given this, we then investigate whether the proposed method may offer improvement over the method of \cite{Blazek2012} in terms of statistical error on $\gamma_{\rm IA}$. 

We test this by computing the signal-to-noise for each method assuming statistical error only (which is dominated by shape noise and / or cosmic variance, depending on scale). In the case of the proposed method, we do so as a function of $a$ and the percentage correlation between the shape-noise of both methods. We then take the ratio of this signal-to-noise for the proposed method vs that of the Blazek et al. 2012 method.

The result is shown in Figure \ref{fig:StoNstat} for the case of both an SDSS type measurment and an LSST+DESI type measurement. We see that in the case of the SDSS measurement, only for a small segment of parameter space does the proposed method `win' over the existing one. The ratio of signal-to-noise quantities is only greater than unity for the extremes of low $a$ (shape-measurement methods producing very different $\gamma_{\rm IA}$ signals and highly correlated noise. However, in the case of LSST+DESI, we see that a much larger portion of the parameter space takes ratio values greater than unity, and in some places the ratio is considerably greater.

The fact that the proposed method gains more in statistical-only signal-to-noise than does the Blazek et al. 2012 method is a result of the expected improvement in photometric redshift uncertainty for LSST + DESI. Both methods of measuring $\gamma_{\rm IA}$ see their signal-to-noise scaling in the same way with the reduced shape noise that is due to improved surface density of sources, number of lenses, etc. However, the methods behave differently as a function of improved photometric redshifts. The statistical-only signal-to-noise of the proposed method is improved by improved photometric redshifts, due to the fact that less galaxies that are not affected by intrinsic alignment will scatter into the sample, therefore $N_{\rm corr}$ is closer to $1$. However, this improved photometric redshift uncertainty does not benefit the Blazek et al. method in the same way. {\bf still need to understand this better}.

\begin{figure*}
\centering
\subfigure{\includegraphics[width=0.45\textwidth]{StoN_2d_stat_SDSS.pdf}}
\subfigure{\includegraphics[width=0.45\textwidth]{StoN_2d_stat_LSST_DESI.pdf}}
\caption{Ratio of statistical-only signal-to-noise for the proposed method vs the method of \cite{Blazek2012}. Left: SDSS. Right: LSST+DESI.}
\label{fig:StoNstat}
\end{figure*}

%-------------------------------------------------------------------------------
\section{Discussion and Conclusions}
\label{sec:conclusion}

\bibliographystyle{h-physrev}
\bibliography{refs}

\end{document}